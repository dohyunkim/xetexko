%% !tex program = xelatex
%%
%% xetexko-doc.tex
%%
%% This file is in the public domain

\documentclass[a4paper]{article}
\usepackage[hmargin=3.8cm]{geometry}
\usepackage[hangul]{xetexko}
\setmainhangulfont{Noto Serif CJK KR}[
  UprightFont=* Light, BoldFont=* Bold,
  Script=Hangul, Language=Korean, AutoFakeSlant,
]
\setsanshangulfont{Noto Sans CJK KR}[
  UprightFont=* DemiLight,
  Script=Hangul, Language=Korean
]
\setmathhangulfont{Noto Sans CJK KR}[
  SizeFeatures={
    {Size=-6,  Font=* Medium},
    {Size=6-9, Font=*},
    {Size=9-,  Font=* DemiLight},
  },
  Script=Hangul, Language=Korean
]
\usepackage{hologo, xcolor, ulem}
\usepackage[bookmarksnumbered,hidelinks]{hyperref}

\edef\verbatim{\unexpanded\expandafter{\verbatim
  \linespread{1.1}\selectfont }}
\def\logoko{\textsf{k}\kern-.2ex\textit{o}}
\def\XeTeX{\hologo{XeTeX}}
\def\xetexko{\XeTeX\kern-.1ex -\logoko}
\def\kotex{\logoko\kern-.1ex .\kern-.1667em \hologo{TeX}}
\pdfstringdefDisableCommands{\def\xetexko{XeTeX-ko}}
\renewcommand*\descriptionlabel[1]{\hspace\labelsep\ttfamily\bfseries
  \color{blue!50!black}#1}
\def\grayhrule{{\color{white!80!black}\hrule height2pt}}
\def\grayvrule{{\color{white!80!black}\vrule width 2pt}}
\newenvironment{plaintex}
  {\par\medskip
    \setbox0=\vbox\bgroup\kern5pt\leftskip15pt \small
    \parindent0pt }
  {\par \kern5pt\egroup
    \dimen0=\dimexpr\pagegoal-\pagetotal\relax
    \ifdim\dimen0<\baselineskip\relax \dimen0\maxdimen \fi
    \ifdim\ht0<\dimen0
      \hbox{\grayvrule \box0}%
    \else
      \setbox2=\vsplit0 to\dimen0
      \lineskip0pt
      \ifvoid2 \else\hbox{\grayvrule \box2}\fi
      \ifvoid0 \else\hbox{\grayvrule \box0}\fi
    \fi \par\smallskip
  }
\newenvironment{example}
  {\hbox\bgroup\grayvrule
    \vbox\bgroup\hsize\dimexpr\textwidth-4pt\relax
    \grayhrule\kern5pt\leftskip10pt\rightskip10pt
  }{\par\kern5pt\grayhrule\egroup\grayvrule\egroup}
\def\cs#1{\texttt{\textbackslash #1}}
\def\hemph#1{\textsf{\bfseries #1}}
\def\hparen#1{{\small\hangulparens(#1)}}
\let\hparens\hparen
\def\keyword#1{\cs{#1}\marginpar{\hfuzz\maxdimen\footnotesize{\color{blue!80}\cs{#1}}}}
\def\vernote#1{v#1\marginpar{\hfuzz\maxdimen\footnotesize{\color{red}☜ v#1}}}

\hangingpunctuations

\begin{document}

\latinparens

\title{\xetexko\ 간단 매뉴얼}
\author{Dohyun Kim \normalsize <\texttt{nomosnomos at gmail com}> \and
  \normalsize <\url{http://github.com/dohyunkim/xetexko}>}
\date{Version 4.4\quad 2025/02/24}
\maketitle

\begin{quote}\small
  For a summary introduction in English,
  please see \verb+README+ file in this package.
\end{quote}

\hangulparens

\section{입력 인코딩}
입력은 원칙적으로 UTF-8으로 한다.
BOM (Byte Order Mark)은 있어도 좋고 없어도 좋다.
EUC-KR이나 CP949 입력도 지원하는데, 다음과 같은 선언이 있으면 된다.
\begin{verbatim}
    \XeTeXinputencoding="korean"
    \XeTeXdefaultencoding="korean"
\end{verbatim}

\section{\XeTeX\ 자체 기능}
\XeTeX 은 자체적으로 유니코드를 잘 지원하므로 \xetexko 를 사용하지
않더라도 한글을 찍는 데 문제가 없다.
\begin{verbatim}
    \usepackage{fontspec}
    \setmainfont{UnBatang}
    \setsansfont{UnDotum}
    \setmonofont{UnTaza}
    \XeTeXlinebreaklocale="ko"
    \XeTeXlinebreakpenalty=50
    \XeTeXlinebreakskip=0pt plus.1ex minus.05ex
\end{verbatim}
하지만 이렇게 사용하면 영문자와 한글이 같은 글꼴로 식자될 뿐만 아니라,
한글 사이 자간이나 영문자와 한글 사이 간격 따위의 \xetexko 가 제공하는
기능을 포기해야 할 것이다.

\section{\xetexko\ 부르기}\label{sec:loading}
\xetexko\ 로드하기: \cs{usepackage\{xetexko\}} 혹은 \cs{usepackage\{kotex\}}.
이 명령으로 fontspec 패키지도 함께 불러들인다.
fontspec에 옵션을 주려면 \xetexko를 부르기 전에 fontspec을 따로 로드하면 된다.

\begin{plaintex}
플레인텍이라면 \verb|\input xetexko.sty| 혹은 \verb|\input kotex.sty|.
\end{plaintex}

패키지 옵션:
\begin{description}\hangulquotes
\item[ hangul ] `제~1~장' `표' `차례' 등 한글 표제를 이용한다.
  절 제목 이하에는 한글 표제가 붙지 않고 숫자로만 표현된다.
  한글이 주된 언어라고 인식하여 줄간격 조정도 수행한다.
\item[ hanja ]  `第~1~章' `表' `目次' 등 한자 표제를 이용한다.
  절 제목 이하에는 한자 표제가 붙지 않고 숫자로만 표현된다.
  한글이 주된 언어라고 인식하여 줄간격 조정도 수행한다.
\end{description}


\section{글꼴 명령}\label{sec:fontcmds}
\begin{verbatim}
    \setmainfont      {Linux Libertine}
    \setmainhangulfont{NanumMyeongjo}
    \setmainhanjafont {UnBatang}
\end{verbatim}
각 명령의 의미는 이름이 뜻하는 그대로이며, fontspec 매뉴얼에 따라
설정한다.  마찬가지로 다음 명령들도 제공한다.
\begin{verbatim}
    \setsansfont \setsanshangulfont \setsanshanjafont
    \setmonofont \setmonohangulfont \setmonohanjafont
\end{verbatim}
main/sans hangul/hanja 폰트에는 따로 \verb|Ligatures=TeX| 옵션을 지시하지
않더라도 자동으로 주어진다.

\hemph{프리앰블}에서 한글 폰트를 지정하지 않으면 영문폰트가 한글에도 적용되고
한자폰트를 지정하지 않으면 한글폰트가 한자에도 적용된다.
영문폰트에 한글·한자 글리프가 없고 한글·한자 폰트도 따로
지시하지 않았다면 \hemph{은바탕/은돋움 트루타입}을 한글·한자 글꼴의 기본값으로
사용한다.

\bigskip

\begin{plaintex}
플레인텍에서도 기본값으로 은바탕을 한글에 사용한다. 한글폰트를 지시하는 방법은 다음과 같다.
\begin{verbatim}
    \font\tenrm="[lmroman10-regular]:mapping=tex-text" at 10pt
    \hangulfont="[UnBatang]:mapping=tex-text" at 10pt
    \hanjafont ="Naver Dictionary:mapping=tex-text" at 10pt
\end{verbatim}
  또한 다음과 같은 새로운 명령을 제공한다.
\begin{verbatim}
    \sethangulfont\untaza ="[UnTaza.ttf]"  at 10pt
    \sethanjafont \undotum="[UnDotum.ttf]" at 10pt
    \expandafter\def\expandafter\tt\expandafter{\tt\untaza\undotum}
\end{verbatim}
  그리하여 예의 \cs{tt} 명령으로 이제 한글, 한자, 영문 글꼴을 동시에 바꿀 수 있게 된다.
\end{plaintex}

\keyword{fallbackhanjafont}를 정의해두면 \hemph{한자 글꼴에도 없는 한자}를
이 글꼴에서 찾아 식자한다. \cs{newfontfamily} 명령 등으로 정의한다.

\section{글꼴 명령 (계속)}
\begin{verbatim}
    \hangulfontspec{NanumGothic}[Ligatures=TeX]
    \hanjafontspec {UnGungseo}[Ligatures=TeX]
\end{verbatim}
임시로 사용할 한글·한자 글꼴을 지시한다.
\cs{adhochangulfont}와 \cs{adhochanjafont}는 이들 명령의 다른 이름이다.

\begin{verbatim}
    \newhangulfontfamily\myhangul{NanumGothic}[Ligatures=TeX]
    \newhanjafontfamily \myhanja {UnGungseo}[Ligatures=TeX]
    \newhangulfontface  \myhangul{NanumGothic Bold}[Ligatures=TeX]
    \newhanjafontface   \myhanja {UnDotumBold}[Ligatures=TeX]
\end{verbatim}
fontspec 패키지의 \cs{setfontfamily} \cs{setfontface} 명령에 해당한다.

\begin{verbatim}
    \addhangulfontfeature{Color=FF0000,InterHangul=-.1em}
    \addhanjafontfeature {Color=00FF00}
\end{verbatim}
fontspec의 \cs{addfontfeature} 명령을 구현한 것이다.
fontspec과 마찬가지로 각 명령의 끝에 `s'를 붙여도 된다.

참고로, 한자와 한글을 같은 글꼴로 식자하고 싶을 때 글꼴 설정을
번거롭게 두 번 할 필요 없이 한글 글꼴만 설정한 뒤 간단히
\keyword{hanjabyhangulfont}를 선언하면 된다.%
\footnote{프리앰블의 main/sans/mono 글꼴 명령들과 달리 새로 지정한 한글 폰트가
  한자에도 자동으로 적용되지는 않는다는 말이다.}


\section{기호 글꼴 명령}\label{sec:symbolfont}
라틴 문자나 CJK 기호를 어떤 폰트로 식자할 것인가를 사용자가 지정할 수 있다.
\begin{verbatim}
    \xetexkofontregime{hangul}[parens=latin, puncts=prevfont]
\end{verbatim}
모든 라틴 문자와 CJK 기호들을 원칙적으로 한글 폰트로 식자하되,
괄호류는 라틴 폰트로 찍고 마침표 따위는 현재 폰트를 따라간다는 의미이다.
명령의 인자에는 latin, hangul, hanja, prevfont 중 하나가 올 수 있어서
각각 라틴 폰트, 한글 폰트, 한자 폰트로 식자하라는 뜻이며
마지막의 prevfont는 \hemph{현재 활성화된 폰트를 그대로 이어받도록} 지시한다.
현재 폰트가 직전 문자의 폰트가 아닐 수 있음에 주의할 것.

옵션 키의 종류와 의미는 다음과 같으며 각 키 값은 인자의 경우와 같다.
\begin{description}\itemsep0pt
\item[alphs] 라틴 알파벳, 숫자, 아래에 해당않는 모든 non-CJK 문자
\item[parens] 라틴 괄호, 중괄호, 대괄호, 꺽쇠, 인용부호 및 그 리거쳐
\item[colons] 라틴 콜론, 세미콜론
\item[hyphens] 하이픈 및 그  리거쳐, 슬래시, en dash, em dash
\item[puncts] 마침표, 물음표, 느낌표, 쉼표
\item[cjksymbols] CJK 구두점, 괄호 및 상징기호
\end{description}

종전에
라틴숫자에 적용되던 \verb|nums| 옵션과
라틴 인용부호에 적용되던 \verb|quotes| 옵션이
v3.0부터는
각각 \verb|alphs| 옵션과 \verb|parens| 옵션에 통합되었음에 유의하라.
또한 en dash, em dash가 종래 \verb|colons|에 속하던 것이 이제
\verb|hyphens|에 속하게 되었다.

기본값은 다음 명령을 준 것과 동일하다.
v3.0부터 괄호류의 기본글꼴이 한글이 되었음도 유의하라.
\begin{verbatim}
    \xetexkofontregime {latin}%
      [ puncts=prevfont, colons=prevfont,
        parens=hangul, cjksymbols=hangul ]
\end{verbatim}
\begin{plaintex}
플레인텍에서는 다음 명령을 사용한다. \LaTeX에서도 쓸 수 있다.
\begin{verbatim}
    \latinalphs      \hangulalphs      \hanjaalphs      \prevfontalphs
    \latinparens     \hangulparens     \hanjaparens     \prevfontparens
    \latincolons     \hangulcolons     \hanjacolons     \prevfontcolons
    \latinhyphens    \hangulhyphens    \hanjahyphens    \prevfonthyphens
    \latinpuncts     \hangulpuncts     \hanjapuncts     \prevfontpuncts
    \latincjksymbols \hangulcjksymbols \hanjacjksymbols \prevfontcjksymbols
\end{verbatim}
다음 명령은 이상의 모든 라틴 문자를 식자할 폰트를 지시한다.
\begin{verbatim}
    \latinmarks      \hangulmarks      \hanjamarks      \prevfontmarks
\end{verbatim}
\end{plaintex}

\section{글꼴 옵션}\label{sec:fontopts}
\hemph{고정폭 글꼴을 제외한} 한글·한자 글꼴 명령에 미세 간격 옵션을
쓸 수 있다. 예컨대
\begin{verbatim}
    \setsanshangulfont{NanumGothic}[InterHangul=-.04em]
\end{verbatim}
이 때 나눔고딕 글꼴에 한글 마이너스 자간이 작동한다.

허용되는 옵션은 다음과 같다.
\begin{description}\itemsep0pt
\item[InterLatinCJK, hu]\quad
  {\addhangulfontfeature{InterLatinCJK=.5em} Latin문자와} 한글 사이 간격을 지시한다.
    기본값 \verb|0.05em|이며
    수식과 CJK 문자 사이에는 이 값의 두 배가 적용된다.
    아래에서 설명할 고문헌조판 모드에서는 기본값이 \verb|0.25em|이며
    수식 직후에 들어가는 값도 동일하다.

\item[InterHangul, interhchar]\quad
  {\addhangulfontfeature{InterHangul=0.3em} 한글자간을} 지시한다.
  기본값은 \verb|0pt|.%
  \footnote{한자 자간 옵션은 제공하지 않는다.
  혹시 한자 자간을 조절하고 싶다면 \cs{XKinterhanjabreak}을 재정의하라.
  참고로 한글 자간을 조절하는 저수준 명령은 \cs{XKinterhangulbreak} 이다.}

\item[CharRaise, charraise]\quad
  글자의 {\addhangulfontfeature{charraise=.5ex}수직 위치}를 조절한다.
  기본값은 \verb|0pt|.%
    \footnote{%
      세로쓰기에서는 이 옵션의 값을 고정길이(가령 pt 단위)로 주어야
      크기가 다른 폰트들 간에 중앙정렬이 이루어진다.
    }
  고정폭 글꼴에서도 이 옵션은 작동한다.
\end{description}
v3.0부터는 다음 옵션들이 아무 효과도 가져오지 않게
변경되었음을 주의하라.
\begin{verbatim}
    lowerperiod         lowerquestion
    lowerexclamation    lowercomma
    preperiodkern       postperiodkern
    prequestionkern     postquestionkern
    preexclamationkern  postexclamationkern
    precommakern        postcommakern
    quoteraise          quotewidth
    postmathskip        PunctRaise
\end{verbatim}

\begin{plaintex}
  플레인텍에서는 다음과 같은 방법으로 제어할 수 있다. 기본값은 \LaTeX에서와 같다.
\begin{verbatim}
    \def\xetexkohu{0.05em }
    \def\xetexkointerhchar{-0.04em }
    \def\xetexkocharraise{-0.1ex }
\end{verbatim}
  이것들을 선언해두고나서 \cs{hangulfont} 명령 등으로 글꼴을
  지정하면 기 정의된 옵션들이 한글폰트 명령에 자동 포함된다.
  한자폰트의 경우도 같다. 따라서 한글폰트 옵션과
  한자폰트 옵션을 서로 달리 지시할 수 있다.
\end{plaintex}

종전에 한글·한자 글꼴의 옵션 기본값을 지정하던
\cs{defaulthangulfontfeatures} 및 \cs{defaulthanjafontfeatures} 명령이
v3.0부터는
아무런 기능을 하지 않게 변경되었음에 유의하라.
폰트옵션의 수가 적어졌기 때문에 이 명령의 유용성도 거의 사라졌다고 판단했다.

\section{그밖의 명령들}\label{sec:etccmds}
\begin{description}
  \item[\cs{everyhangul}, \cs{everyhanja}]
    한글 또는 한자에 대해 글자마다 지시한 명령을 수행한다. 예컨대
    \verb|\everyhangul{\fbox{#1}}| 명령은 모든
    {\everyhangul{\fbox{#1}}\fboxsep=1pt 한글 주위에 박스}를
    두른다.

  \item[\cs{disablekoreanfonts}]
    한글폰트를 따로 설정하지 않고 영문 폰트를 따라간다.
    이 때  \cs{setmainfont} 따위가 한글과 한자에도 적용된다.

  \item[\cs{disablecjksymbolspacing}]
    CJK 구두점의 폭을 \verb|0.5em|으로 강제설정하는 것을 방지한다.
    \xetexko는 낡은 고정폭 한글 폰트들을 고려하여 이 기능을 구현하였으나
    현대적인 한글 폰트는 이러한 과잉친절이 오히려 성가실 경우가 있다.

    \cs{nocompresspunctuations}는 이 명령과 동의어이다.\\
    반대로
    \keyword{compresspunctuations}는 반각을 강제하라는 명령이다.
    이것이 기본값이므로 특별한 경우가 아니면 따로 선언할 필요가 없다.

  \item[\cs{disablehangulspacing}]
    한글조판에 관련된 모든 미세간격을 영(zero)으로 돌린다. 한글 자간,
    영문자와 한글 사이 간격, 구두점과 한글^^b7한자 사이 간격 등이 작동하지 않는다.
    그러나 CJK 글자 사이의 줄바꿈 허용만은 유지된다. 또한
    \verb|CharRaise|는 자간과 무관하므로 여전히 작동한다.

  \item[\cs{disablehangulspacingandlinebreak}]
    위 명령과 마찬가지로 모든 간격을 금지할 뿐만 아니라 CJK 글자 사이의
    줄바꿈까지도 먹지 않게 한다.
    일부 고급 오픈타입 기능을 사용하기 위해서 이 명령이 필요할 때가 있다.
    이를테면, 폰트가 지원하는 한글 리거쳐가 동작하지 않을 때 이 명령을
    선언하면 문제가 해결된다.%
    \footnote{%
      이 명령은 그룹 안에서 실행해야 할 것이다. 그렇더라도
      \cs{enablehangulspacingandlinebreak} 명령을 그룹을 닫고나서 선언해야
      원래 기능이 완전히 복구된다.
      }

\end{description}
v3.0부터는 종전의
\cs{disableautojosa} 및 \cs{disablejamoautojosa} 명령이 아무 기능도 하지 않게
변경되었음을 유의하라.

\section{옛한글}

옛한글을 사용하기 위해선 \verb|Script=Hangul| 옵션을 주는 것으로 족하다.
옛한글 식자는 고급 오픈타입 기능을 이용하므로
폰트가 이런 기능을 제공해야만 가능한 일이다.\footnote{%
이런 기능을 제공하는 폰트로는 윈도즈~8에 번들된 맑은고딕이
있으며 공개폰트로는 은바탕, 함초롬~LVT, Source Han Sans (본고딕),
Source Han Serif (본명조), Noto Sans CJK, Noto Serif CJK,
나눔명조옛한글, 나눔바른고딕옛한글, KoPubWorld 바탕, KoPubWorld 돋움 등이 있다.
또한
\textsf{^^^^1112^^^^119e^^^^11ab글}에 번들된 함초롬 글꼴에도
최근 옛한글 지원이 추가되었다.}

\begin{figure}
  \begin{example}
    \sffamily \obeylines\leftskip10em
뎨 가ᄂᆞᆫ 뎌 각시 본 듯도 ᄒᆞᆫ뎌이고.
天텬上상白ᄇᆡᆨ玉옥京경을 엇디ᄒᆞ야 離니別별ᄒᆞ고
ᄒᆡ 다 져믄 날의 눌을 보라 가시ᄂᆞᆫ고
\medbreak
어와 네여이고 내 ᄉᆞ셜 드러보오.
내 얼굴 이 거동이 님 괴얌즉 ᄒᆞᆫ가마ᄂᆞᆫ
엇딘디 날 보시고 네로다 녀기실ᄉᆡ
나도 님을 미더 군 ᄠᅥ디 전혀 업서
이ᄅᆡ야 교ᄐᆡ야 어ᄌᆞ러이 구돗ᄯᅥᆫ디
반기시ᄂᆞᆫ ᄂᆞᆺ비치 녜와 엇디 다ᄅᆞ신고.
  \end{example}
  \caption{옛한글 조판 보기}\label{fig:yethangul}
\end{figure}

\begin{verbatim}
    \setmainhangulfont{Malgun Gothic}[ Script=Hangul ]
\end{verbatim}
\begin{plaintex}
플레인텍이라면, \cs{hangulfont="Malgun Gothic:script=hang" at 10pt}.
\end{plaintex}

\section{세로쓰기}\label{sec:verttype}

세로쓰기를 위한 글꼴 지정은 다음과 같다.
역시 고급 오픈타입 기능을 이용하는 것이므로 폰트가 이를 지원해야 한다.
\begin{verbatim}
    \newhangulfontfamily\vertfont{Noto Sans CJK KR}
      [Vertical=Alternates, RawFeature=vertical]
\end{verbatim}

문서 전체를 세로쓰기할 때 \keyword{verticaltypesetting}을 선언하면
면주는 그대로 두고 본문만 세로쓰기한다.\footnote{면주의 글꼴에는
  \texttt{vertical} 옵션이 없어야 할 것이다.
  면주 글꼴을 따로 지정하는 것은 사용자의 몫이다. }

v3.6부터 \verb|verticaltypesetting| 환경을 이용하여
몇몇 페이지만 세로쓰기로 조판할 수 있다.
역시 면주는 가로로 조판되며, 환경의 앞뒤에서 쪽나눔이 행해진다.

문서의 일부를 세로쓰기 하려면 \verb|vertical| 환경을 이용한다.
vertical 환경에는 세로쓰기 박스의 높이를 인자로 준다.
v3.6부터는 만약 인자가 비어있거나 \cs{empty}이면
본문을 한 줄짜리 박스로 식자하도록 했다. 이때 높이는 natural height가 된다.

\begin{figure}
\begin{example}
\hfil\begin{vertical}{18em}
  \linespread{1.5}\sffamily
  \addhangulfontfeatures{Vertical=Alternates,CharRaise=.5ex,RawFeature=vertical}
  \hanjabyhangulfont
  \typesetclassic \latinparens
  \parindent0pt \everypar{\hangindent1em \hangafter1 }
  \spaceskip=.5em plus .25em minus .125em
  \quad 님의 침묵 (The Silent Beloved)
\smallbreak
\hfil\hfil 한 용 운\hfil
\bigbreak
님은 갓슴니다 아아 사랑하는나의님은 갓슴니다\par
푸른산빗을깨치고 단풍나무숩을향하야난 적은길을 거러서 참어떨치고 갓슴니다\par
黃金의꽃가티 굿고빗나든 옛盟誓는 차듸찬띠끌이되야서 한숨의 微風에 나러갓슴니다\par
날카로은 첫〈키쓰〉의追憶은 나의運命의指針을 돌너노코 뒷거름처서 사러젓슴니다\par
\hellipsis\par
아아 님은갓지마는 나는 님을보내지 아니하얏슴니다\par
제곡조를못이기는 사랑의노래는 님의沈默을 휩싸고돔니다\par
\end{vertical}\hfil
\end{example}
  \caption{\linespread{1.2}\selectfont
  세로쓰기 보기.
  Noto Sans CJK KR 글꼴을 이용했다. 이 글꼴의 세로쓰기 어간이 1em으로
  매우 커서 \cs{spaceskip=0.5em plus.25em minus.125em}을 강제했다.
  \texttt{CharRaise=.5ex}도 글꼴 옵션에 추가했다.
  또한 \cs{latinparens}를 선언했는데
  제목 다음의 괄호가 CJK 괄호가 아니라 아스키 괄호이기 때문이다.
  }\label{fig:vertical}
\end{figure}

\begin{verbatim}
    \begin{vertical}{12em}
      \vertfont
      \hanjabyhangulfont
      세로쓰기로 植字할 부분
    \end{vertical}
\end{verbatim}

혹시 세로로 식자되지 않는 라틴문자 부분이 있다면
베이스라인이 한글과 어울리지 않을 수 있다. 이를 교정하려면
\verb|CharRaise| 옵션을 한글 폰트에 준다.\footnote{%
  세로쓰기에서는 일반적으로 \texttt{CharRaise} 옵션을 주는 쪽이 권장된다.
  이 방법이 여의치 않으면 라틴문자 부분을 \cs{vertlatin} 명령으로 감쌀 수도
  있는데, 이때 인자가 박스로 처리되어 줄바꿈이 일어나지 않는다.}

세로쓰기 도중에 문서의 일부를 가로쓰기로 삽입하려면
\verb|horizontal| 환경을 이용한다.
horizontal 환경에는 가로쓰기 박스의 너비를 인자로 준다.
v3.6부터는 만약 인자가 비어있거나 \cs{empty}이면
본문을 한 줄짜리 박스로 식자하도록 했다. 이때 너비는 당연히 natural width가 된다.

\begin{plaintex}
  플레인텍에서 문서 전체를 세로쓰기하려면 문서 첫머리에 \cs{verticaltypesetting}.

  몇몇 페이지만 세로로 쓰려면 \cs{beginverticaltypesetting} \ldots\ \cs{endverticaltypesetting}.

  문서 일부를 세로쓰기하려면 가령,
\begin{verbatim}
    \vertical{}
      \hangulfont="Noto Sans CJK KR:vertical;+vert"
      \hanjabyhangulfont
      세로쓰기로 植字할 부분
    \endvertical
\end{verbatim}

  세로쓰기 중에 잠시 가로로 식자하려면 \cs{horizontal\{\}} \ldots\ \cs{endhorizontal}.
\end{plaintex}

\section{고문헌 조판}

\verb|typesetclassic| 환경은 고문헌 조판에 적합한 환경이다.

불필요한 공백을 없애주는 기능은
v3.0부터는 자동으로 동작하지 않는다.  사용자가
\keyword{removeclassicspaces} 명령을 주어야 CJK 글자간 공백을
제거하기 시작한다.

간단한 일본어, 중국어 문단 정도를 식자하는 데도 이 환경을 이용할 수 있으나
각 언어의 전용 패키지들에 비해 기능이 많이 모자란다.\footnote{%
  그러나 \texttt{japanese}, \texttt{Schinese}, \texttt{Tchinese} 환경이 여전히 제공된다.
  \texttt{chinese}는 \texttt{Schinese}의 다른 이름이다.
  특히 중국어 폰트를 사용할 때는 \texttt{Schinese}나 \texttt{Tchinese}를 명시하는 게 좋다.
  }

\begin{plaintex}
  플레인텍에서는 다음 방식으로 사용한다.
\begin{verbatim}
    \begingroup \typesetclassic
      ...
    \endgroup
\end{verbatim}
\end{plaintex}


\keyword{inhibitglue} 명령은
해당 지점에 자동 삽입되는 공백을 영(zero)으로 만들어준다.

현대문서 조판에는 \verb|typesetmodern| 환경이 쓰인다. 이 환경은 기본값이므로
특별한 경우가 아니면 따로 선언할 필요가 없다.

\def\analect{%
  \removeclassicspaces
  子曰:「學而時習之,不亦說乎?有朋自遠方來,不亦樂乎?人不知而不慍,不亦君子乎?」\par
  有子曰:「其爲人也 孝弟,而好犯上者,鮮矣!不好犯上,而好作亂者,未之有也!
  君子務本,本立而道生;孝弟也者,其爲仁之本與?」\par
  子曰:「巧言令色,鮮矣仁!」\par
  曾子曰:「吾日三省吾身:爲人謀,而不忠乎?與朋友交,而不信乎?傳,不習乎?」\par
  子曰:「道千乘之國,敬事而信,節用而愛人,使民以時。」\par
  子曰:「弟子入則孝,出則弟;謹而信,汎愛衆;而親仁,行有餘力,則以學文。」
}
\begin{figure}
  \begin{example}\linespread{1.3}
    \sffamily\typesetclassic
    \analect
  \end{example}
\caption{\linespread{1.2}\selectfont
  고문헌 조판 보기.
  typesetclassic 환경을 이용했다.
  \cs{removeclassicspaces}도 선언했다.
  }\label{fig:ancientdoc}
\end{figure}

\begin{figure}
  \begin{example}\linespread{1.3}
    \hfil\begin{vertical}{19em}
      \linespread{1.5}\sffamily\typesetclassic
      \addhangulfontfeatures{Vertical=Alternates,CharRaise=.5ex,RawFeature=vertical}
      \hanjabyhangulfont \parindent0pt \everypar{\hangindent1em \hangafter1 }
      \analect
    \end{vertical}\hfil
  \end{example}
  \caption{\linespread{1.2}\selectfont
  고문헌 세로쓰기 조판 보기.
  vertical 환경을 이용했다.
  \cs{hangingpunctuations}가 선언되어 있어서 구두점이
  행 끝에 걸린 것을 볼 수 있다.
  }\label{fig:ancientdocvertical}
\end{figure}

\section{매달린 구두점}
\hemph{하이픈을 제외한} 문장부호들를 판면 바깥에 식자하여 행 끝에 매달리게 만들 수 있다.

v3.0부터는 \keyword{hangingpunctuations} 명령으로 활성화한다.
\cs{hangingpunctuation}은 이 명령과 동의어이다.
당분간 xetexko-hanging.sty을 로드하는 것으로도 활성화되도록 해 두었다.
이렇게 활성화되면 기본값으로 온점, 반점, 고리점, 모점에 대해서 전부 내밀기가 작동한다.

v3.2부터는 사용자가 원하는 문장부호를 원하는 만큼 판면 밖으로 내밀 수 있다.
\begin{verbatim}
    \sethangingratio"28=0.5 %
    \sethangingratio"3002=1 %
\end{verbatim}
등호 앞의 숫자는 유니코드 코드포인트이고 뒤의 숫자는 글자폭에 대한 비율을 뜻한다.
따라서 위 예는 여는괄호(U+0028)를 절반만 내밀고 고리점(U+3002)은 완전히 밀어내라는 지시다.
어느 방향으로 내밀지는 \xetexko 가 알아서 판단한다.
이러한 지시를 취소하려면,
\begin{verbatim}
    \unsethangingratio"28
    \unsethangingratio"3002
\end{verbatim}

microtype 패키지의 protrusion 기능이 잘 작동하지 않는 경우가 있어,%
\footnote{%
  원시명령 \cs{lpcode} \cs{rpcode} 등 엔진이 제공하는 기능을 말한다.
  이들 원시명령에는 글리프 인덱스를 지시해야 하는데 어떤 글자의 글리프 인덱스를
  알아내는 일은 간단치가 않다.
  \XeTeX\ reference guide를 참고하라.
  }
어쩔 수 없이 제공하게 되었으므로 예외적으로만 사용할 것을 권한다.

%\begin{figure}
%\Large
%\hangingpunctuation
%\setbox0\hbox{줄 끝에 매달린 구두점}
%\vbox{\hfil\hbox{\vrule\vbox{\hsize\wd0
%  줄 끝에 매달린 구두점.
%  줄 끝에 매달린 구두점,
%  줄 끝에 매달린 구두점、
%  줄 끝에 매달린 구두점。
%  줄 끝에 매달린 구두점
%  }\vrule}\hfil}
%\caption{기본값이 적용된 매달린 구두점}\label{fig:hanging}
%\end{figure}

\section{수식 한글}
수식 모드에서도 한글을 \hparens{hbox로 감싸지 않고} 직접 입력할 수 있다.
\begin{quote}
  \verb|$가^{나^다}$|\quad$\Rightarrow\quad가^{나^다}$
\end{quote}
v3.0부터는
자동으로 수식한글을 잡아주지 않으므로 사용자가 다음과 같이 명시적으로 선언해야 한다.
\begin{verbatim}
    \setmathhangulfont{Noto Sans CJK KR}[
      SizeFeatures={ {Size=-6,  Font=* Medium},
                     {Size=6-9, Font=*},
                     {Size=9-,  Font=* DemiLight} }
    ]
\end{verbatim}
\begin{plaintex}
  플레인텍에서는 \cs{mathhangulfont="[UnBatang.ttf]"}와 같이 설정한다.

  또는 좀 더 그럴듯하게 하려면:\par
  \leftskip=3em\noindent
\begin{verbatim}
    \font\texthangul="Noto Sans CJK KR DemiLight" at 10pt
    \font\scripthangul="Noto Sans CJK KR Regular" at 7pt
    \font\scriptscripthangul="Noto Sans CJK KR Medium" at 5pt
    \setmathhangulfonts\texthangul\scripthangul\scriptscripthangul
\end{verbatim}
\end{plaintex}

현재 한글만 쓸 수 있게 설정되어 있다.
한자도 수식에 직접 입력하려면 사용자는
\begin{verbatim}
    \setmathhangulblock{4E00}{9FC3}
\end{verbatim}
명령으로 유니코드 블럭을 추가 지정해야 한다.

\section{자동조사}
자동조사는 \kotex 의 경우와 대체로 동일하게 동작한다.
게다가 문장 중에서도 동작하고
플레인텍에서까지 사용할 수 있다.
그럼에도 불구하고 사용자가 임의로 제어할 필요가 생긴다면
\cs{jong} \cs{jung} \cs{rieul} 중 하나를
미리 선언해 둘 수 있다. 이를테면,
\begin{verbatim}
    \def\mypage{페이지\jung}
    \def\mypage{쪽\jong}
    \def\mystar{별\rieul}
\end{verbatim}

\section{드러냄표}
\keyword{dotemph} 명령을 사용하는
\dotemph{드러냄표} 강조도 \kotex 과 대동소이하다.
기본은 글자 위에 점을 찍는 형태이나 다음 명령을 이용해 개인적으로 선호하는
기호를 드러냄표로 쓸 수 있다.
\begin{verbatim}
    \def\dotemphraise{0.4em }
    \def\dotemphchar{\bfseries ^^^^02d9}
\end{verbatim}
전자는 드러냄표를 피강조 글자 위로 끌어올리는 길이를,
후자는 드러냄표 기호 자체를 정의한다.
\verb|^^^^02d9|는 유니코드 코드포인트를 뜻하는 16진수이고
숫자 대신 직접 문자를 입력해도 된다.
플레인텍에서도 드러냄표 강조를 사용할 수 있다.

\section{고아글자 문제}
어느 정도는 고아글자가 생기지 않게 만들어 두었으나,
\XeTeX 에서는 고아글자를 자동으로 완전히 회피하기는 어렵다.
필요하다면 사용자가 적당한 곳에 \cs{nobreak} 명령을 삽입해야 한다.

\section{한국어 번호}
항목 번호를 한국어 기호로 붙일 수 있다. \kotex과 마찬가지로
\cs{jaso} \cs{gana} \cs{ojaso} \cs{ogana} \cs{pjaso}
\cs{pgana} \cs{onum} \cs{pnum} \cs{oeng} \cs{peng}
\cs{hnum} \cs{Hnum} \cs{hroman} \cs{hRoman} \cs{hNum}
\cs{hanjanum}
따위를 사용한다.

\section{밑줄긋기}
\LaTeX 에서는 \uwave{ulem 패키지}를 로드하는 것만으로 자동 지원된다.

\cs{xetexkoulemsupport} 명령이 v3.3부터 부활하였으므로
플레인텍에서는 이 명령을 명시해야 할 경우가 있을 것이다.%
\footnote{%
  ulem 명령어의 인자 안에서도 \cs{dotemph} 명령을 쓸 수 있다.
}


%\hfill \fboxsep=-\fboxrule \fbox{\vbox to1em{\hbox to1em{\hss}\vss}}

\bigskip
\centerline{\hfil\vrule width.5\textwidth height.4pt \relax\hfil}

\section*{주요 변경 이력}

\begin{small}\linespread{1.2}\selectfont
\begin{description}\itemsep=0pt
    \item[v4.1] 2022-03-18
      \begin{itemize}
        \item U+223C \hparen{\texttt{^^^^223c}}를
          \verb|cjksymbols|에 할당함으로써
          한글 폰트로 찍히는 게 기본값이 되었다.
      \end{itemize}
    \item[v4.0] 2021-09-06
      \begin{itemize}
        \item \xetexko 가 제공하는 대부분의 기능은 폰트 인코딩이
          유니코드 인코딩(\texttt{TU})이거나 플레인텍처럼 폰트 인코딩 개념이 없는
          경우에만 동작한다.
      \end{itemize}
    \item[v3.6] 2021-06-10
      \begin{itemize}
        \item \hyperref[sec:verttype]
          {\cs{verticaltypesetting}을 환경으로도 쓸 수 있다}.
        \item \hyperref[sec:verttype]
          {\texttt{vertical} 및 \texttt{horizontal} 환경의 인자를 비워둘 수 있다}.
      \end{itemize}
    \item[v3.4] 2021-05-15
      \begin{itemize}
        \item \hyperref[sec:fontcmds]
          {글꼴 전환 방식의 내부적 변화 및 그에 따른 플레인텍 한글 폰트 명령의 의미 변경}
      \end{itemize}
    \item[v3.3] 2021-04-20
      \begin{itemize}
        \item \cs{xetexkoulemsupport} 명령 부활
        \item `㉾' 등 한글기호를 \verb=cjksymbols=에 할당
      \end{itemize}
    \item[v3.2] 2021-04-08
      \begin{itemize}
        \item 매달린 구두점이 기본값으로 따옴표에는 작동하지 않고 마침표^^b7쉼표에만 작동한다.
        \item \cs{sethangingratio}, \cs{unsethangingratio} 명령 추가
      \end{itemize}
    \item[v3.1] 2021-03-22
      \begin{itemize}
        \item 플레인텍 명령 \cs{sethangulfont}, \cs{sethanjafont} 추가
      \end{itemize}
    \item[v3.0] 2021-03-20
      \begin{itemize}
        \item \verb|typesetclassic|, \verb|typesetmodern| 환경이 추가되었다.
        \item \cs{removeclassicspaces} 명령이 추가되었다.
        \item \cs{compresspunctuations}, \cs{nocompresspunctuations} 명령이 추가되었다.
        \item \cs{hangingpunctuations} 명령이 추가되었다.
        \item 디폴트 고정폭 한글 폰트의 메트릭을 Latin Modern Mono에
          맞추어 조정하였다.
        \item 글꼴옵션이 \verb|InterLatinCJK|, \verb|InterHangul|, \verb|CharRaise|
          외에는 모두 무력화되었다.
        \item \cs{xetexkofontregime}의 키값 \verb|nums|와 \verb|quotes|가
          각각 \verb|alphs|와 \verb|parens|에 통합되었다.
        \item \cs{xetexkoulemsupport} 명령이 무력화되었다.
        \item \cs{disableautojosa}, \cs{disablejamoautojosa} 명령이 무력화되었다.
        \item \cs{defaulthangulfontfeatures} \cs{defaulthanjafontfeatures}
          명령이 무력화되었다.
        \item 수식한글을 자동으로 잡아주지 않는다.
      \end{itemize}
    \item[v2.23] 2020-02-17
      \begin{itemize}
        \item \hyperref[sec:fontcmds]
          {디폴트 한글 폰트가 나눔글꼴에서 은글꼴로 변경되었다.}
      \end{itemize}
    \item[v2.13] 2015-06-16
      \begin{itemize}
        \item \hyperref[sec:verttype]
          {\texttt{horizontal} 환경이 추가되었다.}
      \end{itemize}
    \item[v2.10] 2015-01-16
      \begin{itemize}
        \item 패키지 옵션 \texttt{[unfonts]}를 더는 지원하지 않는다. 이 옵션을
          사용한 기존 문서는 기본 글꼴인 은글꼴 트루타입으로 식자될 것이다.
      \end{itemize}
    \item[v2.7] 2014-06-17
      \begin{itemize}
        \item \hyperref[sec:fontcmds]
          {main/sans hangul/hanja 글꼴에는 디폴트 옵션
          \texttt{Ligatures=TeX}이 주어진다.}
        \item \hyperref[sec:fontopts]
          {\cs{defaulthangulfontfeatures}와 \cs{defaulthanjafontfeatures}
          명령이 생겼다.}
      \end{itemize}
    \item[v2.4] 2014-02-02
      \begin{itemize}
        \item Variation Selector를 이용한 한자 변형 글리프 접근을 허용한다.
      \end{itemize}
    \item[v2.2] 2013-12-06
      \begin{itemize}
        \item \hyperref[sec:fontopts]
          {글꼴 옵션 \texttt{CharRaise}를 추가했다.}
        \item \hyperref[sec:fontcmds]
          {\cs{fallbackhanjafont} 명령을 인식한다.}
        \item \hyperref[sec:etccmds]
          {\cs{everyhangul}, \cs{everyhanja} 명령이 추가되었다.}
        \item \hyperref[sec:symbolfont]
          {하이픈 문자를 \hparen{직전 글꼴이 아니라} 라틴 글꼴에서 찾는
          것이 기본값이 되었다.}
      \end{itemize}
    \item[v2.0] 2013-03-27
      \begin{itemize}
        \item \hyperref[sec:loading]
          {플레인텍에서 xetexko.tex이 아니라 xetexko.sty를 불러야 한다.}
        \item \hyperref[sec:symbolfont]
          {CJK 기호문자를 \hparen{한자가 아니라} 한글 글꼴에서 찾는 것이
          기본값이 되었다.}
        \item \hyperref[sec:fontopts]
          {\texttt{interpunctskern} 글꼴 옵션을 없애고 \texttt{quoteraise}를
          추가했다.}
      \end{itemize}
  \end{description}
\end{small}

\end{document}
