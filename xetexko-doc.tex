%% !tex program = xelatex
%%
%% xetexko-doc.tex
%%
%% This file is in the public domain

\documentclass[a4paper]{article}
\usepackage[hmargin=3.8cm]{geometry}
\usepackage[hangul]{xetexko}
\setmainhangulfont[BoldFont=KoPubBatang_Pro Bold.otf,
  Ligatures=TeX]{KoPubBatang_Pro Light.otf}
\setsanshangulfont[BoldFont=KoPubDotum_Pro Bold.otf,
  Ligatures=TeX]{KoPubDotum_Pro Medium.otf}
\usepackage{xetexko-hanging}
\usepackage{hologo,xcolor,ulem}
\usepackage[bookmarksnumbered]{hyperref}

\edef\verbatim{\unexpanded\expandafter{\verbatim
  \linespread{1.1}\selectfont }}
\def\logoko{\textsf{k}\kern-.2ex\textit{o}}
\def\XeTeX{\hologo{XeTeX}}
\def\xetexko{\XeTeX\kern-.1ex -\logoko}
\def\kotex{\logoko\kern-.1ex .\kern-.1667em \hologo{TeX}}
\pdfstringdefDisableCommands{\def\xetexko{XeTeX-ko}}
\renewcommand*\descriptionlabel[1]{\hspace\labelsep\ttfamily\bfseries
  \color{blue!50!black}#1}
\def\grayhrule{{\color{white!80!black}\hrule height2pt}}
\def\grayvrule{{\color{white!80!black}\vrule width 2pt}}
\newenvironment{plaintex}
  {\setbox0=\vbox\bgroup\hsize=\dimexpr\textwidth-12pt\relax\kern10pt\par\small}
  {\par\vskip-\lastskip\kern10pt\egroup \par\medskip\par
    \dimen0=\dimexpr\pagegoal-\pagetotal\relax
    \ifdim\ht0<\dimen0
      \hbox{\grayvrule\kern10pt\box0}%
    \else
      \lineskip=0pt \splittopskip=10pt
      \setbox2=\vsplit0 to\dimen0
      \ifvoid2 \else\hbox{\grayvrule\kern10pt\box2}\fi
      \ifvoid0 \else\hbox{\grayvrule\kern10pt\box0}\fi
    \fi \par\medskip\par }
\newenvironment{example}
  {\hbox\bgroup\grayvrule
    \vbox\bgroup\hsize\dimexpr\textwidth-4pt\relax
    \grayhrule\kern1em\leftskip1em\rightskip1em
  }{\par\kern1em\grayhrule\egroup\grayvrule\egroup}
\def\cs#1{\texttt{\textbackslash #1}}
\def\hemph#1{\textsf{\bfseries #1}}
\def\hparen#1{{\small\hangulparens(#1)}}
\def\hparens#1{\leavevmode\unskip---{\small #1}---\ignorespaces}
\parindent0pt
\begin{document}
\title{\xetexko\ 간단 매뉴얼}
\author{Dohyun Kim \normalsize <\texttt{nomos at ktug org}> \and
  \normalsize <\url{http://github.com/dohyunkim/xetexko}>}
\date{Version 2.5\quad\today}
\maketitle

\begin{quote}\small
  For a summary introduction in English,
  please see \verb+README+ file in this package.

  \begin{itemize}\itemsep=0pt
    \item[v2.0]---
    \item 플레인텍에서 xetexko.tex이 아니라 xetexko.sty를 불러야 한다.
    \item \verb+[unfonts]+ 옵션이 없으면 은글꼴을 자동으로 부르지 않는다.
    \item CJK 기호문자를 \hparen{한자가 아니라} 한글 글꼴에서 찾는 것이
      기본값이 되었다.
    \item \verb|interpunctskern| 글꼴 옵션을 없애고 \verb|quoteraise|를
      추가했다.
    \item[v2.2]---
    \item 글꼴 옵션 \verb|charraise|를 추가했다.
    \item \verb|\fallbackhanjafont| 명령을 인식한다.
    \item \verb|\everyhangul|, \verb|\everyhanja| 명령이 추가되었다.
    \item 하이픈 문자를 \hparen{직전 글꼴이 아니라} 라틴 글꼴에서 찾는
      것이 기본값이 되었다.
    \item[v2.4]---
    \item Variation Selector를 이용한 한자 변형 글리프 접근을 허용한다.
  \end{itemize}
\end{quote}

\section{입력 인코딩}
입력은 원칙적으로 UTF-8으로 한다.
BOM (Byte Order Mark)은 있어도 좋고 없어도 좋다.
EUC-KR이나 CP949 입력도 지원하는데, 다음과 같은 선언이 있으면 된다.
\begin{verbatim}
  \XeTeXinputencoding="korean"
  \XeTeXdefaultencoding="korean"
\end{verbatim}

\section{\XeTeX\ 자체 기능}
\XeTeX 은 자체적으로 유니코드를 지원하므로 \xetexko 를 사용하지
않아도 한글을 찍을 수 있다.
\begin{verbatim}
  \usepackage{fontspec}
  \setmainfont[Ligatures=TeX]{UnBatang}
  \setsansfont[Ligatures=TeX]{UnDotum}
  \setmonofont{UnTaza}
  \XeTeXlinebreaklocale="ko"
  \XeTeXlinebreakpenalty=50
  \XeTeXlinebreakskip=0pt plus.1ex minus.05ex
\end{verbatim}
하지만 이렇게 사용하면 영문자와 한글이 같은 글꼴로 식자될 뿐만 아니라,
한글 사이 자간이나 영문자와 한글 사이 간격 따위의 \xetexko 가 제공하는
기능을 포기해야 할 것이다.

\section{\xetexko\ 부르기}
\xetexko\ 로드하기: \cs{usepackage\{xetexko\}}.

이 명령은 fontspec 및 xunicode 패키지도 불러온다.
종전과 달리 xltxtra 패키지를 함께 부르지 않음을 주의할 것.

\begin{plaintex}
플레인텍이라면 \verb|\input xetexko.sty| 명령으로 로드한다.
\end{plaintex}

패키지 옵션:
\begin{description}
\item[ hangul ] `제1장' `표' `차례' 등 한글 표제를 이용한다.
  절 제목 이하에는 한글 표제가 붙지 않고 숫자로만 표현된다.
  한글이 주된 언어라고 인식하여 줄간격 조정도 수행한다.
\item[ hanja ]  `第1章' `表' `目次' 등 한자 표제를 이용한다.
  절 제목 이하에는 한자 표제가 붙지 않고 숫자로만 표현된다.
  한글이 주된 언어라고 인식하여 줄간격 조정도 수행한다.
\item[ unfonts ] 은글꼴을 불러온다. \hemph{이 옵션이 없으면
  사용자가 한글 글꼴을 지정해야} 할 수 있다.
\end{description}


\section{글꼴 명령}
\begin{verbatim}
  \setmainfont      [Ligatures=TeX]{Linux Libertine}
  \setmainhangulfont[Ligatures=TeX]{NanumMyeongjo}
  \setmainhanjafont [Ligatures=TeX]{UnBatang}
\end{verbatim}
각 명령의 의미는 이름이 뜻하는 그대로이며, fontspec 매뉴얼에 따라
설정한다. 한글폰트를 따로 지정하지 않으면 영문폰트가 한글에도 사용된다.
한자폰트를 따로 지정하지 않으면 한글폰트가 한자에도 적용된다.

마찬가지로 다음 명령들도 제공한다.
\begin{verbatim}
  \setsansfont \setsanshangulfont \setsanshanjafont
  \setmonofont \setmonohangulfont \setmonohanjafont
\end{verbatim}

\begin{plaintex}
플레인텍이라면 다음처럼 한다.
\begin{verbatim}
    \font\tenrm="[lmroman10-regular]:mapping=tex-text" at 10pt
    \hangulfont="[UnBatang]:mapping=tex-text" at 10pt
    \hanjafont ="Naver Dictionary:mapping=tex-text" at 10pt
\end{verbatim}
\end{plaintex}

한편 버전 2.2부터는 \cs{fallbackhanjafont}를 정의해두면
\hemph{한자 글꼴에도 없는 한자}를
이 글꼴에서 찾아 식자한다. \verb|\newfontfamily| 명령으로 정의한다.

\section{글꼴 명령 (계속)}
\begin{verbatim}
  \hangulfontspec[Ligatures=TeX]{NanumGothic}
  \hanjafontspec [Ligatures=TeX]{UnGungseo}
\end{verbatim}
임시로 사용할 한글·한자 글꼴을 지시한다.
\cs{adhochangulfont}와 \cs{adhochanjafont}는 이들 명령의 다른 이름이다.

\begin{verbatim}
  \newhangulfontfamily\myhangul[Ligatures=TeX]{NanumGothic}
  \newhanjafontfamily \myhanja [Ligatures=TeX]{UnGungseo}
  \newhangulfontface  \myhangul[Ligatures=TeX]{NanumGothic Bold}
  \newhanjafontface   \myhanja [Ligatures=TeX]{UnDotumBold}
\end{verbatim}
fontspec 패키지의 \verb+\newfontfamily+ \verb+\newfontface+ 명령을
수정한 것이다.

\begin{verbatim}
  \addhangulfontfeature{Color=FF0000,interhchar=-.1em}
  \addhanjafontfeature {Color=00FF00}
\end{verbatim}
fontspec의 \verb+\addfontfeature+ 명령을 구현한 것이다.
fontspec과 마찬가지로 각 명령의 끝에 `s'를 붙여도 된다.

참고로, 한자와 한글을 같은 글꼴로 식자하고 싶을 때 글꼴 설정을
번거롭게 두 번 할 필요 없이 한글 글꼴만 설정한 뒤 간단히
  \cs{hanjabyhangulfont}를 선언하면 된다.


\section{기호 글꼴 명령}
라틴 문자나 CJK 기호를 어떤 폰트로 식자할 것인가를 사용자가 지정할 수 있다.
\begin{verbatim}
  \xetexkofontregime[quotes=latin, puncts=prevfont]{hangul}
\end{verbatim}
모든 라틴 문자와 CJK 기호들을 원칙적으로 한글 폰트로 식자하되,
따옴표는 라틴 폰트로 찍고 마침표 따위는 현재 폰트를 따라간다는 의미이다.
옵션은 생략할 수 있다.

명령의 인자에는 latin, hangul, hanja, prevfont 중 하나가 올 수 있어서
각각 라틴 폰트, 한글 폰트, 한자 폰트로 식자하라는 뜻이며
마지막의 prevfont는 \hemph{현재 활성화된 폰트를 그대로 받도록} 지시한다.
현재 폰트가 직전 문자의 폰트가 아닐 수 있음에 주의할 것.

옵션 키의 종류와 의미는 다음과 같으며 각 키 값은 인자와 마찬가지로
latin, hangul, hanja, prevfont 가운데 하나를 허용한다.
\begin{description}\itemsep0pt\hangulparens
\item[alphs] 라틴 알파벳 및 아래에 해당하지 않는 모든 라틴 문자
\item[nums] 라틴 숫자 \hparen{수식 제외}
\item[parens] 라틴 괄호, 중괄호, 대괄호, 꺽쇠
\item[quotes] accent grave, apostrophy 및 이들의 리거쳐
\item[colons] 라틴 콜론, 세미콜론, endash, emdash
\item[hyphens] 하이픈, 하이픈으로 이루어지는 리거쳐, 슬래시
\item[puncts] 마침표, 물음표, 느낌표, 쉼표
\item[cjksymbols] CJK 구두점, 괄호 및 상징기호
\end{description}

기본값은 다음 명령을 준 것과 동일하다. \hemph{CJK 기호가 한글폰트}로,
\hemph{하이픈은 라틴폰트}로
식자되게 기본값이 변경되었음에 유의할 것.
\begin{verbatim}
  \xetexkofontregime
  [ puncts=prevfont, colons=prevfont, cjksymbols=hangul ]
  {latin}
\end{verbatim}

\begin{plaintex}
플레인텍에서는 다음 명령을 사용한다. \LaTeX에서도 쓸 수 있다.
\begin{verbatim}
  \latinalphs      \hangulalphs      \hanjaalphs      \prevfontalphs
  \latinnums       \hangulnums       \hanjanums       \prevfontnums
  \latinparens     \hangulparens     \hanjaparens     \prevfontparens
  \latinquotes     \hangulquotes     \hanjaquotes     \prevfontquotes
  \latinhyphens    \hangulhyphens    \hanjahyphens    \prevfonthyphens
  \latincolons     \hangulcolons     \hanjacolons     \prevfontcolons
  \latinpuncts     \hangulpuncts     \hanjapuncts     \prevfontpuncts
  \latincjksymbols \hangulcjksymbols \hanjacjksymbols \prevfontcjksymbols
\end{verbatim}
다음 명령은 이상의 모든 라틴 문자를 식자할 폰트를 지시한다.
\begin{verbatim}
  \latinmarks      \hangulmarks      \hanjamarks      \prevfontmarks
\end{verbatim}
\end{plaintex}

\section{간격 조정}
한글 관련 미세 간격 조정은 다음과 같이 폰트 옵션으로 지시한다.
\begin{verbatim}
  \setsanshangulfont[interhchar=-.04em]{NanumGothic}
\end{verbatim}
이 때 나눔고딕 글꼴에만 한글 마이너스 자간이 작동한다.
\dotemph{고정폭 글꼴을 제외한} 한글·한자 글꼴
명령에 이런 옵션을 쓸 수 있다.
허용되는 옵션은 다음과 같다. 각 항목의 길이값은 디폴트값을 뜻한다.
\begin{description}\itemsep0pt
\item[hu]
  영문자와 한글 사이 간격. 영문 괄호와 한글 사이 간격은 그 두 배이다. 0.06em.
\item[interhchar]
  {\addhangulfontfeature{interhchar=0.3em} 한글 자간.}%
  \footnote{한자 자간 옵션은 제공하지 않는다.
  혹시 한자 자간을 조절하고 싶다면 \cs{XKinterhanjabreak}을 재정의하라.
  참고로 한글 자간을 조절하는 저수준 명령은 \cs{XKinterhangulbreak} 이다.}
  0pt.
\item[charraise] 버전 2.2의 새로운 옵션으로
  글자의 \hemph{\addhangulfontfeature{charraise=.5ex}수직 위치}를 조절한다.
  다른 글꼴 관련 효과와 충돌할 수 있어 주의를 요한다. 0pt.
\item[lowerperiod]
  CJK 문자 직후의 마침표를 아래로 끌어내리는
  {\addhangulfontfeature{lowerperiod=.5ex}정도. 0pt.}
\item[lowerquestion]
  CJK 문자 직후의 물음표를 아래로 끌어내리는 정도. 0pt.
\item[lowerexclamation]
  CJK 문자 직후의 느낌표를 아래로 끌어내리는 정도. 0pt.
\item[lowercomma]
  CJK 문자 직후의 쉼표를 아래로 끌어내리는 정도. 0pt.
\item[preperiodkern]
  {\addhangulfontfeature{preperiodkern=1ex}CJK 문자 직후의 마침표 앞에
  들어가는 간격. 0pt.}
\item[postperiodkern]
  {\addhangulfontfeature{postperiodkern=1ex}CJK 문자 직후의 마침표 뒤에
  들어가는 간격. 0pt.}
\item[prequestionkern]
  CJK 문자 직후의 물음표 앞에 들어가는 간격. 0pt.
\item[postquestionkern]
  CJK 문자 직후의 물음표 뒤에 들어가는 간격. 0pt.
\item[preexclamationkern]
  CJK 문자 직후의 느낌표 앞에 들어가는 간격. 0pt.
\item[postexclamationkern]
  CJK 문자 직후의 느낌표 뒤에 들어가는 간격. 0pt.
\item[precommakern]
  CJK 문자 직후의 쉼표 앞에 들어가는 간격. 0pt.
\item[postcommakern]
  CJK 문자 직후의 쉼표 뒤에 들어가는 간격. 0pt.
\item[quoteraise]
  영문 인용부호\hparens{아스키문자로 입력하는 것을 말함}를
  위로 끌어올리는 정도. 0pt.
\item[quotewidth]
  영문 인용부호의 폭. 기본값은 natural width.
\item[postmathskip]
  수식과 CJK 문자 사이의 간격.  기본값은 위 hu 값의 두 배.
\end{description}

\begin{plaintex}
플레인텍에서는 다음과 같은 방법으로 제어할 수 있다. 기본값은 \LaTeX에서와 같다.
\begin{verbatim}
  \def\xetexkohu{0.06em }
  \def\xetexkointerhchar{-0.04em }
  \def\xetexkocharraise{-0.1ex }
  \def\xetexkolowerperiod{0.15ex }
  \def\xetexkolowerquestion{0.15ex }
  \def\xetexkolowerexclamation{0.15ex }
  \def\xetexkolowercomma{0.15ex }
  \def\xetexkopreperiodkern {-0.1em }
  \def\xetexkopostperiodkern{}
  \def\xetexkoprequestionkern {0.2em }
  \def\xetexkopostquestionkern{0.2em }
  \def\xetexkopreexclamationkern {0.1em }
  \def\xetexkopostexclamationkern{0.1em }
  \def\xetexkoprecommakern {0.1em }
  \def\xetexkopostcommakern{0.1em }
  \def\xetexkoquoteraise{0.1em }
  \def\xetexkoquotewidth{0.33em }
  \def\xetexkopostmathskip{0.2em }
\end{verbatim}
\end{plaintex}

\section{그밖의 명령들}
\begin{description}
  \item[\cs{everyhangul}, \cs{everyhanja}]
    한글 또는 한자에 대해 글자마다 지시한 명령을 수행한다. 예컨대
    \verb|\everyhangul{\fbox{#1}}| 명령은 모든
    {\everyhangul{\fbox{#1}}\fboxsep=1pt 한글 주위에 박스}를
    두른다.  2.2 버전의 새로운 명령. 다른 글꼴 옵션과 충돌할 수 있으므로
    주의를 요한다.

  \item[\cs{disablekoreanfonts}]
한글폰트를 따로 설정하지 않고 영문 폰트를 따라간다.
이 때  \verb|\setmainfont| 따위가 한글과 한자에도 적용된다.

  \item[\cs{disableautojosa}]
혹시 있을지 모를 예기치 못한 기능간 충돌에 대비하여 자동조사 기능을
끄는 명령을 제공한다.

  \item[\cs{disablejamoautojosa}]
옛한글 자모에 한해서 자동조사 기능을 끈다.
옛한글 자모에 영향을 주는 \verb+Mapping+ 글꼴 옵션을 사용해야 한다면
이 명령을 선언해두어야 정상 작동한다.
버전 2.1에서 새롭게 추가된 명령이다.

  \item[\cs{disablecjksymbolspacing}]
CJK 구두점의 폭을 0.5em으로 강제설정하는 것을 방지한다.
\xetexko는 낡은 고정폭 한글 폰트들을 고려하여 이 기능을 구현하였으나
현대적인 한글 폰트는 이러한 과잉친절이 오히려 성가실 경우가 있다.

  \item[\cs{disablehangulspacing}]
한글조판에 관련된 모든 미세간격을 영(zero)으로 돌린다. 한글 자간도,
영문자와 한글 사이 간격도, 구두점 위치 조정도, 구두점 사이의 간격도
모두 작동하지 않는다. 그러나 CJK 글자 사이의 줄바꿈 허용만은 유지된다.

  \item[\cs{disablehangulspacingandlinebreak}]
위 명령과 마찬가지로 모든 미세간격을 금지할 뿐만 아니라 CJK 글자 사이의
줄바꿈까지도 먹지 않게 한다.
일부 고급 오픈타입 기능을 사용하기 위해서 이 명령이 필요할 때가 있다.
이를테면, interhchar 옵션에 의해 자간을 주면 한글 리거쳐가
동작하지 않게 되거니와 이 명령을 선언하면 문제가
해결된다.\footnote{만약 사용자가
  {\ttfamily\string\XeTeXlinebreaklocale}도 선언했다면
  {\ttfamily\string\XeTeXlinebreaklocale""} 명령도 요구된다.
  이 원시명령은 그룹 내에서 쓰더라도 항상 전역적인 효과를 가짐을 유의하라}
\end{description}

\section{옛한글}

\begin{figure}
  \begin{example}
    \hangulfontspec[YetHangul]{HCR Dotum LVT}
    \hanjabyhangulfont%\parindent1em
    \obeylines\leftskip10em
뎨 가ᄂᆞᆫ 뎌 각시 본 듯도 ᄒᆞᆫ뎌이고.
天텬上상白ᄇᆡᆨ玉옥京경을 엇디ᄒᆞ야 離니別별ᄒᆞ고
ᄒᆡ 다 져믄 날의 눌을 보라 가시ᄂᆞᆫ고
\medbreak
어와 네여이고 내 ᄉᆞ셜 드러보오.
내 얼굴 이 거동이 님 괴얌즉 ᄒᆞᆫ가마ᄂᆞᆫ
엇딘디 날 보시고 네로다 녀기실ᄉᆡ
나도 님을 미더 군 ᄠᅥ디 전혀 업서
이ᄅᆡ야 교ᄐᆡ야 어ᄌᆞ러이 구돗ᄯᅥᆫ디
반기시ᄂᆞᆫ ᄂᆞᆺ비치 녜와 엇디 다ᄅᆞ신고.
  \end{example}
  \caption{옛한글 조판 보기}\label{fig:yethangul}
\end{figure}

옛한글을 사용하기 위해선 \verb|Script=Hangul| 옵션을 주는 것으로 족하다.
옛한글 식자는 고급 오픈타입 기능을 이용하므로
폰트가 이런 기능을 제공해야만 가능한 일이다.\footnote{%
\today\ 현재, 이런 기능을 제공하는 폰트로는 마이크로소프트 오피스 플러스팩이
제공하는 옛한글 글꼴과 윈도즈~8에 번들된 맑은고딕이 있으며
공개폰트로는 은바탕과 KTUG판 함초롬 글꼴이 있다.}
\begin{verbatim}
  \setmainhangulfont[ Script=Hangul ]{Malgun Gothic}
\end{verbatim}
\begin{plaintex}
플레인텍이라면,
\begin{verbatim}
  \hangulfont="Malgun Gothic:script=hang" at 10pt
\end{verbatim}
\end{plaintex}

\section{세로쓰기}

\begin{figure}
\begin{example}
  \verticalem=1.05em % hcr fonts' vertical em
\hfil\begin{vertical}{16\verticalem}
  \sffamily % hcr fonts do not have vrt2 feature
  \hangulfontspec[RawFeature=vertical;+vert;+vhal,charraise=.5ex]
    {HCR Dotum LVT}
  \hanjabyhangulfont \parindent-\verticalem\leftskip\verticalem
  \def\hellipsis{$\cdots\cdots$} % hcr fonts' ellipsis not good
\noindent 님의 침묵 (The Silent Beloved)
\smallbreak
\hfil\hfil 한 용 운\hfil
\bigbreak
님은 갓슴니다 아아 사랑하는나의님은 갓슴니다\par
푸른산빗을깨치고 단풍나무숩을향하야난 적은길을 거러서 참어떨치고 갓슴니다\par
黃金의꽃가티 굿고빗나든 옛盟誓는 차듸찬띠끌이되야서 한숨의 微風에 나러갓슴니다\par
날카로은 첫〈키쓰〉의追憶은 나의運命의指針을 돌너노코 뒷거름처서 사러젓슴니다\par
\hellipsis\par
아아 님은갓지마는 나는 님을보내지 아니하얏슴니다\par
제곡조를못이기는 사랑의노래는 님의沈默을 휩싸고돔니다\par
\end{vertical}\hfil
\end{example}
\caption{세로쓰기 보기}\label{fig:vertical}
\end{figure}

세로쓰기를 위한 글꼴 지정은 다음과 같다.
역시 고급 오픈타입 기능을 이용하는 것이므로
폰트가 이를 지원해야 한다.
\begin{verbatim}
  \setmainhangulfont[ Vertical=RotatedGlyphs ]{Adobe Myungjo Std}
\end{verbatim}
문서 전체를 세로쓰기할 때 \cs{verticaltypesetting}을 선언하면
면주는 그대로 두고 본문만 세로쓰기한다.\footnote{당연히 면주의 글꼴은
\texttt{Vertical} 옵션이 없는 것이라야 할 것이다.
면주 글꼴을 따로 지정하는 것은 사용자의 몫이다. }
문서의 일부를 세로쓰기 하려면 vertical 환경을 이용한다.
vertical 환경의 인자는 세로쓰기 박스의 세로 길이이다.
보통 그 뒤에  세로쓰기에 사용할 글꼴 명령이 온다.
\begin{verbatim}
  \begin{vertical}{12em}
    \hangulfontspec[ Vertical=RotatedGlyphs ]{Adobe Myungjo Std}
    세로쓰기로 식자할 부분
  \end{vertical}
\end{verbatim}
만일 세로쓰기의 em 박스 크기가 가로쓰기와 다르다면
\cs{verticalem=1.05em}과 같은 선언을 미리 해두는 것이 좋다.

혹시 세로로 식자되지 않는 라틴문자 부분이 있다면
베이스라인이 한글과 어울리지 않을 수 있다. 이를 교정하려면
\verb|charraise| 옵션을 한글 폰트에 준다.
이 방법이 여의치 않으면 라틴문자 부분을 \cs{vertlatin} 명령으로 감쌀 수도
있는데, 이때 인자가 박스로 처리되어 그 안에서 줄바꿈이 일어나지 않는다.
\medbreak
\begin{plaintex}
플레인텍이라면,
\begin{verbatim}
  \vertical{16em}\hangulfont="Adobe Myungjo Std:vertical"
    세로쓰기로 식자할 부분
  \endvertical
\end{verbatim}
\end{plaintex}

\section{일본어·중국어·고문헌}

\begin{figure}
  \begin{example}
    \hanjafontspec{STFangsong}\chinese
子曰:「學而時習之,不亦說乎?有朋自遠方來,不亦樂乎?人不知而不慍,不亦君子乎?」

有子曰:「其為人也孝弟,而好犯上者,鮮矣!不好犯上,而好作亂者,未之有也!君子務本,本立而道生;孝弟也者,其為仁之本歟?」

子曰:「巧言令色,鮮矣仁!」

曾子曰:「吾日三省吾身:為人謀,而不忠乎?與朋友交,而不信乎?傳,不習乎?」
  \end{example}
\caption{고문헌 조판 보기}\label{fig:ancientdoc}
\end{figure}

띄어쓰기가 없는 옛문헌이나 일본어 단락 따위를 삽입할 때
japanese 혹은 chinese 환경을 선언하는 것이 좋다.
들여쓰기\footnote {들여쓰기 기본값은 japanese는 1em 이고 chinese는 2em 이다.}
및 자간 조절이 자동으로 이루어지며
사용자가 입력한 불필요한 공백도 없애준다.
Schinese 환경은 chinese의 다른 이름이다.
Tchinese도 제공하는데 중국어 번체 글꼴을 사용한다면 이 환경을 이용하라.

\cs{inhibitglue} 명령은 일본어 등 환경에서 가끔 유용한데,
해당 지점에 자동 삽입되는 공백을 영(zero)으로
만들어준다.

\section{매달린 구두점}
마침표, 쉼표, 인용부호를 판면 바깥에 식자하여 행 끝에 매달리게 만든다.
\begin{verbatim}
  \usepackage{xetexko-hanging}
\end{verbatim}
\cs{hangingpunctuation} 명령은 버전 2.2 이후 불필요해졌다.
글자 내밀기 기본값에 차이가 좀 있긴 하지만,
microtype 패키지를 로드하는 것으로도 유사한 결과를 얻을 수 있다.

%\begin{figure}
%\Large
%\hangingpunctuation
%\setbox0\hbox{줄 끝에 매달린 구두점}
%\vbox{\hfil\hbox{\vrule\vbox{\hsize\wd0
%  줄 끝에 매달린 구두점.
%  줄 끝에 매달린 구두점,
%  ``줄 끝에 매달린 구두점''
%  `줄 끝에 매달린 구두점'
%  “줄 끝에 매달린 구두점”
%  ‘줄 끝에 매달린 구두점’
%  줄 끝에 매달린 구두점、
%  줄 끝에 매달린 구두점。
%  줄 끝에 매달린 구두점
%  }\vrule}\hfil}
%\caption{매달린 구두점 보기}\label{fig:hanging}
%\end{figure}

\section{수식 한글}
수식 모드에서도 한글을 \hparens{hbox로 감싸지 않고} 직접 입력할 수 있다.
\begin{quote}
  \verb|$가^{나^다}$|\quad$\Rightarrow\quad가^{나^다}$
\end{quote}
수식 한글은 따로 설정하지 않으면 문서의 기본 글꼴을 따라가지만
굳이 설정하려면 다음과 같이 한다. 이 명령은 플레인텍에서도 쓸 수 있다.
\begin{verbatim}
  \setmathhangulfont{NanumGothic}
\end{verbatim}
현재 한글만 쓸 수 있게 설정되어 있다.
한자도 수식에 직접 입력하려면 사용자는
\begin{verbatim}
  \setmathhangulblock{4E00}{9FC3}
\end{verbatim}
명령으로 유니코드 블럭을 추가 지정해야 한다.

\section{자동조사}
자동조사는 \kotex 의 경우와 기본적으로 동일하게 동작한다.
게다가 문장 중에서도 동작하고
플레인텍에서까지 사용할 수 있다.
그럼에도 불구하고 사용자가 임의로 제어할 필요가 생긴다면
\cs{jong} \cs{jung} \cs{rieul} 중 하나를
미리 선언해 둘 수 있다. 이를테면,
\begin{verbatim}
  \def\mypage{페이지\jung}
  \def\mypage{쪽\jong}
  \def\mystar{별\rieul}
\end{verbatim}

\section{드러냄표}
\dotemph{드러냄표} 강조도 \kotex 과 대동소이하다.
기본은 글자 위에 점을 찍는 형태이나 다음 명령을 이용해 개인적으로 선호하는
기호를 드러냄표로 쓸 수 있다.
\begin{verbatim}
  \def\dotemphraise{0.4em }
  \def\dotemphchar{\bfseries ^^^^02d9}
\end{verbatim}
전자는 드러냄표를 피강조 글자 위로 끌어올리는 길이를,
후자는 드러냄표 기호 자체를 정의한다.
\verb|^^^^02d9|는 유니코드 코드포인트를 뜻하는 16진수이고
숫자 대신 직접 문자를 입력해도 된다.
플레인텍에서도 드러냄표 강조를 사용할 수 있다.

\section{고아글자 문제}
어느 정도는 고아글자가 생기지 않게 만들어 두었으나,
\XeTeX 에서는 고아글자를 자동으로 완전히 회피하기는 어렵다.
필요하다면 사용자가 적당한 곳에 \verb|\nobreak| 명령을 삽입해야 한다.

\section{한국어 번호}
항목 번호를 한국어 기호로 붙일 수 있다. \kotex과 마찬가지로
\verb|\jaso| \verb|\gana| \verb|\ojaso| \verb|\ogana| \verb|\pjaso|
\verb|\pgana| \verb|\onum| \verb|\pnum| \verb|\oeng| \verb|\peng|
\verb|\hnum| \verb|\Hnum| \verb|\hroman| \verb|\hRoman| \verb|\hNum|
\verb|\hanjanum|
따위를 사용한다.

\section{밑줄긋기}
\LaTeX에서는 \uline{ulem 패키지}를 로드하는 것만으로 자동 지원된다.
\begin{plaintex}
플레인텍에서는 패키지 로드 후 사용자가 \cs{xetexkoulemsupport}
명령을 내려주어야 한다.
\end{plaintex}
\hfill \fboxsep=-\fboxrule \fbox{\vbox to1em{\hbox to1em{\hss}\vss}}


\end{document}
